%%%%%%%%%%%%%%%%%
% Resume by Kobie Arndt
%%%%%%%%%%%%%%%%

%% If you are using \orcid or academicons
%% icons, make sure you have the academicons
%% option here, and compile with XeLaTeX
%% or LuaLaTeX.
% \documentclass[10pt,a4paper,academicons]{altacv}

%% Use the "normalphoto" option if you want a normal photo instead of cropped to a circle
% \documentclass[10pt,a4paper,normalphoto]{altacv}

\documentclass[10pt,a4paper,ragged2e]{altacv}

%% AltaCV uses the fontawesome and academicon fonts
%% and packages.
%% See texdoc.net/pkg/fontawecome and http://texdoc.net/pkg/academicons for full list of symbols. You MUST compile with XeLaTeX or LuaLaTeX if you want to use academicons.

% Change the page layout if you need to
\geometry{left=1cm,right=9cm,marginparwidth=6.8cm,marginparsep=1.2cm,top=1.25cm,bottom=1.25cm}

% Change the font if you want to, depending on whether
% you're using pdflatex or xelatex/lualatex
\ifxetexorluatex
  % If using xelatex or lualatex:
  \setmainfont{Carlito}
\else
  % If using pdflatex:
  \usepackage[utf8]{inputenc}
  \usepackage[T1]{fontenc}
  \usepackage[default]{lato}
  \usepackage[none]{hyphenat}
  \usepackage{hyperref}
\fi

% Change the colours if you want to
\definecolor{VividPurple}{HTML}{3E0097}
\definecolor{SlateGrey}{HTML}{2E2E2E}
\definecolor{LightGrey}{HTML}{666666}
\colorlet{heading}{VividPurple}
\colorlet{accent}{VividPurple}
\colorlet{emphasis}{SlateGrey}
\colorlet{body}{LightGrey}

% Change the bullets for itemize and rating marker
% for \cvskill if you want to
\renewcommand{\itemmarker}{{\small\textbullet}}
\renewcommand{\ratingmarker}{\faCircle}

%% sample.bib contains your publications
\addbibresource{sample.bib}

\begin{document}
\name{Kobie Arndt}
\tagline{Seeking an Internship for Summer 2020 and or Fall 2020 }

\personalinfo{%
  % Not all of these are required!
  % You can add your own with \printinfo{symbol}{detail}
    \normalsize
      \email{\href{mailto:arndtkobie@gmail.com}{arndtkobie@gmail.com}}
        \phone{1(570)-657-5707}
          \linkedin{\href{linkedin.com/in/kobie-arndt-5a6a17171/}{kobie-arndt}}
            \github{\href{github.com/Kobie127}{Kobie127}}
   
   

%   \orcid{orcid.org/0000-0000-0000-0000} % Obviously making this up too. If you want to use this field (and also other academicons symbols), add "academicons" option to \documentclass{altacv}
}


%% Make the header extend all the way to the right, if you want.
\begin{adjustwidth}{}{-8cm}
\makecvheader
\end{adjustwidth}


%% Depending on your tastes, you may want to make fonts of itemize environments slightly smaller
\AtBeginEnvironment{itemize}{\small}

%% Provide the file name containing the sidebar contents as an optional parameter to \cvsection.
%% You can always just use \marginpar{...} if you do
%% not need to align the top of the contents to any
%% \cvsection title in the "main" bar.
\cvsection[page1sidebar]{Experience}

\cvevent{Website Developer Freelancer}{Digital Adhesion}{May 2019 - August 2019}{www.digitaladhesion.com}
\begin{itemize}
\item  Designed and developed a website for a national digital marketing agency.
\item The website was built using Bootstrap, CSS3, JavaScript, and PHP.  The site also uses Google API's such as reCaptcha and Google Analytics.
\item A Digital Dictionary was built, which is an area on the site that has the most comprehensive of all digital marketing terms using JavaScript.

\end{itemize}



\cvsection{Projects}
\cvevent{Imagine RIT LED Music Light Visualizer}{Arduino, C++}{February 2019 -- April 2019}{github.com/Kobie127/MusicVisualizer}
\begin{itemize}
\item This project was made for the smart car that was built by the Computer Science House for ImagineRIT.
\item It uses the spectrum shield board to read seven bands of equalization from music input.
\item The user is then able to select what frequency of bass they want to listen to via the visualizer.
\end{itemize}
\divider

\cvevent{Math Parser}{C}{Oct 2019 -- Nov 2019}{}
\begin{itemize}
    \item This project is able to parse a postfix expression math operation and compute the answer.
    \item The program is also able to parse a symbol table and get the values associated with the variable names and can compute the answer based off the mathematical operation.
    \item The program uses stacks, tree nodes, pointers, and memory allocation.
\end{itemize}
\divider

\cvevent{E-Commerce Web App}{React, Redux, Hooks, GraphQL, ContextAPI, Stripe, Firebase}{Sept 2019 -- Present}{}
\begin{itemize}
    \item This project is currently in development but the goal of it is to learn how to use the said technologies and to have the web app working 
    like any other e-commerce app by having the user be able to purchase items.
\end{itemize}
\divider

\cvevent{Christmas Shopping Helper}{Java}{Nov 2019 -- Present}{github.com/Kobie127/ChristmasHelper}
\begin{itemize}
\item This project takes in the user's desired spending value and prints out all combinations of items they can buy from someone's Christmas list.
\item The program parses a .txt file and adds all values into a HashMap data structure.
\item The program is also able to parse a .pdf file and then takes the values and converts it into a .txt file.
\end{itemize}




% \cvevent{Technical Administrator \& Project Coordinator}{Landmark Group}{May 2011 -- August 2013}{Dubai, UAE}
% \begin{itemize}
% \item Managing the internal \& hosted network infrastructure including firewalls, servers, switches and telephony
% \item Managing the IT budget with cost effective methods
% \end{itemize}

% \divider

% \cvevent{Product Manager \& UI Lead}{Google}{Oct 2001 -- July 2005}{Palo Alto, CA}

% \begin{itemize}
% \item Appointed by the founder Larry Page in 2001 to lead the Product Management and User Interaction teams
% \item Optimized Google's homepage and A/B tested every minor detail to increase usability (incl.~spacing between words, % color schemes and pixel-by-pixel element alignment)
% \end{itemize}
%
% \divider

% \cvevent{Product Engineer}{Google}{23 June 1999 -- 2001}{Palo Alto, CA}

% \begin{itemize}
% \item Joined the company as employe \#20 and female employee \#1
% \item Developed targeted advertisement in order to use user's search queries and show them related ads
% \end{itemize}


\clearpage

% \cvsection[page2sidebar]{Publications}
% 
% \nocite{*}
% 
% \printbibliography[heading=pubtype,title={\printinfo{\faBook}{Books}},type=book]
% 
% \divider
% 
% \printbibliography[heading=pubtype,title={\printinfo{\faFileTextO}{Journal Articles}}, type=article]
% 
% \divider
% 
% \printbibliography[heading=pubtype,title={\printinfo{\faGroup}{Conference Proceedings}},type=inproceedings]
% 
%% If the NEXT page doesn't start with a \cvsection but you'd
%% still like to add a sidebar, then use this command on THIS
%% page to add it. The optional argument lets you pull up the
%% sidebar a bit so that it looks aligned with the top of the
%% main column.
% \addnextpagesidebar[-1ex]{page3sidebar}


\end{document}
