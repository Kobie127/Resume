%!TEX TS-program = xelatex

% use only XeLaTeX or LuaLaTeX to compile
\documentclass[a4paper]{comcv}

\usepackage[english]{babel}

\title{Kobie Arndt Resume}
\fullname{Kobie}{Arndt}{}
\cvtitle{A Computer Science student seeking an internship for Summer 2021.}
\begin
\phone{\href{tel: 15706575707}{(570) 657-5707}} 
\email{arndtkobie@gmail.com}
\github{https://github.com/Kobie127}{Kobie127}
\linkedin{https://www.linkedin.com/in/kobie-arndt-5a6a17171/}{Kobie Arndt}

% \phone{(732) 850-5097}


\begin{document}

\section{Education}
\combosection{Rochester Institute of Technology}{B.S. in Computer Science}{August 2018 - May 2023}{}

\section{Experience}

\combosection{Red Argyle}{Software Engineering Intern}{August 2020 - Present}{\vspace{\topsep}
    \begin{tightlist}
        \item Refactored a document viewing component on the Salesforce platform from the Aura framework into the newer \newline Lightning Web Component framework, improving the overall performance. 
        \smallskip
        \item Implemented new features to the document viewing component including sharing documents across the platform, adding a new viewing option for the user, and allowing tagging of documents which  can be referenced across the \newline platform using Apex and Lightning Web Components.
        \smallskip
        \item Developed a host of new automated features to help boost the workflow for the user in Apex and Lightning \newline Web Components.
        \smallskip
        \item Followed the test driven development pattern by writing unit tests to ensure that all Apex classes functioned as intended.
        \item Wrote technical documentation and drew state machines which outlined all new and existing processes on the project.
        
    \end{tightlist}
}

\vspace{\topsep}

\combosection{Athlete Studio}{Full Stack Engineer Intern}{June 2020 - August 2020}{\vspace{\topsep}}
    \begin{tightlist}
        \item Helped refactor the back-end infrastructure from proprietary software to using AWS Lambda, API Gateway, and \newline DynamoDB.  
        \item Followed a test driven development pattern by writing tests for the API's written in AWS Lambda using Postman software. 
        \item Developed websites for professional athletes that reached thousands of users using HTML, SCSS, JQuery, Handlebars.js, and interfaced with the Shopify API.
    \end{tightlist}

\vspace{\topsep}


\section{Projects}
\combosection{E-Commerce Web App}{React}{May 2020 – Present}{\href{https://github.com/Kobie127/E-Commerce-App}{github.com/Kobie127/E-Commerce-App}\vspace{\topsep}}
\smallskip
\begin{tightlist}
    \item This is currently in development.
    \item The web app uses the React component design to display all of the items in the system but will be later refactored into Redux.
    \item The project also uses Google’s Firebase to warehouse the data and the Stripe API to handle payments.
\end{tightlist}

\combosection{2D Platformer}{Unity, C\#}{March 2020 – May 2020}{\href{https://github.com/Kobie127/2D-Platformer}{github.com/Kobie127/2D-Platformer}\vspace{\topsep}}
\smallskip
\begin{tightlist}
    \item 2D platformer game that is built in Unity and C\#.
    \item  The game has a multitude of features such as enemy AI, saving and loading systems, achievements based on game progression, and multiple player controllers based on if the player is in game or the level select screen.
    \item The game also features a boss battle which is built as a state machine and is dictated by how the player interacts with it.
\end{tightlist}


\combosection{Imagine RIT LED Music Light Visualizer}{Arduino, C++}{February 2019 – April 2019}{\href{https://github.com/Kobie127/MusicVisualizer}{github.com/Kobie127/MusicVisualizer}\vspace{\topsep}}
\smallskip
    \begin{tightlist}
        \item This project was made for the smart car that was built by the Computer Science House for ImagineRIT.
        \item  It uses the spectrum shield board to read seven bands of equalization from music input.
        \item The user is then able to select what frequency of bass they want to listen to via the visualizer.
    \end{tightlist}

\section{Skills}
\begin{itemize}
    \item {\bf{Languages: }}  {Java, Apex, Python, C\#, HTML 5, SCSS/CSS, JavaScript/JQuery, Node.js, C++, C} % Skills
    \item {\bf{Technologies: }} {SQL, DynamoDB, Firebase, Lightning Web Components, React, AWS Lambda, AWS, API Gateway,\newline Postman, Salesforce, Git, GitHub, Unity, Arduino}
\end{itemize}


\section{Extracurriculars}
\combosection{Computer Science House}{Member, Social Director, History Director}{August 2018 – Present}{\href{https://csh.rit.edu/}{csh.rit.edu}\vspace{\topsep}}
\smallskip
\begin{tightlist}
    \item Computer Science House is a living and learning community with a helpful environment that emphasizes hands-on learning and projects outside of the classroom.

    \item  The History director is in charge of keeping contact with alumni and making sure house history is preserved.

    \item The Social director is in charge of organizing, funding, and running most social events for the organization’s members.

\end{tightlist}



\end{document}
